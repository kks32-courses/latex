
\documentclass[times]{article}

%% 1. Uncomment the packages.
%\usepackage{amsfonts} % Math fonts
%\usepackage{amsmath}  % Align environment
%\usepackage{amssymb}  % Math symbols
%\usepackage{mathptmx} % Math times font
%\usepackage{cleveref} % Intelligent cross-referencing

\title{Article with lots of equations}
\author{Krishna Kumar}
\date{}

\begin{document}

\maketitle

% 2. Align the following inline equation with mathmode $$. Use ^ for superscript
% and _ for subscript
You can also do an inline equation of (a+b)2 = a2 + b2 + 2ab.

% 3. Make sure in-line equations are within the Mathmode, i.e., enclosed within $$
Another inline equation is the Euler's equation: e^{i\pi}=-1. This beautiful equation connects three major constants of mathematics, Euler's Number \textit{e}, the ratio of the circumference of a circle to its diameter, pi, and the square root of -1, i.e., \textit{i}.

% 4. Cross reference the equations. Use~\cref{eq:unique_label}
I can refer to the Schr\"{o}dinger's equation %(cross-reference the equation).
Also the one with multiple equations and a single number is %(cross-reference the equation).

% 5. It's time to have multiple equations and align them. Note: & aligns the 
% equations and see use of \nonumber. Uncomment the following section and see 
% how it affects the output. 

% 6. Try using [fleqn] as a document class option to see what happens then add
% & on left hand side of the equals sign on all equations and see the output like y & = ax+ b

%\begin{align}
%	y   & =  ax+b \nonumber\\
%	y+1 & = ax+(b+1)\\
%	    & = ax+(b+2)-1
%\end{align}
%
%
%\begin{align}
%\label{eq:yequation}
%\begin{aligned}
%	y   & =  ax+b \\
%	y+1 & = ax+(b+1)\\
%	    & = ax+(b+2)-1
%\end{aligned}
%\end{align}
%
%
%% Try using gather environment to center the equations
%
%\begin{gather}
%	y     =  ax+b \nonumber\\
%	y+1   =  ax+(b+1)\\
%	      =  ax+(b+2)-1
%\end{gather}

\end{document}
