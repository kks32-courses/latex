\documentclass[10pt,times]{beamer}
\usepackage{amsfonts}
\usepackage{amsmath}
\usepackage{amssymb}
\usepackage{mathptmx}

\usepackage{color}
\usepackage{minted}
\usepackage{hyperref}
\usepackage{multicol}
\usepackage{tabularx}
\usepackage{booktabs}
\usepackage{menukeys}

% Stolen from John Miller's LaTeX course
\newcommand{\bftt}[1]{\textbf{\texttt{#1}}}
\newcommand{\comment}[1]{{\color[HTML]{008080}\textit{\textbf{\texttt{#1}}}}}
\newcommand{\cmmd}[1]{{\color[HTML]{008000}\bftt{#1}}}
\newcommand{\bs}{$\backslash$}
\newcommand{\cmdbs}[1]{\cmmd{\bs#1}}
\newcommand{\lcb}{\char '173}
\newcommand{\rcb}{\char '175}
\newcommand{\cmdbegin}[1]{\cmdbs{begin\lcb}\bftt{#1}\cmmd{\rcb}}
\newcommand{\cmdend}[1]{\cmdbs{end\lcb}\bftt{#1}\cmmd{\rcb}}



%\newcommand{\wllogo}{\textbf{write\textrm{\LaTeX}}}

% this is where the example source files are loaded from
% do not include a trailing slash
\newcommand{\wllogo}{\textbf{Overleaf}}
\newcommand{\fileuri}{https://raw.githubusercontent.com/kks32/latex-course/master/exercises/}
\newcommand{\wlserver}{https://www.overleaf.com}
\newcommand{\wlnewdoc}[1]{\wlserver/docs?snip\_uri=\fileuri#1\&splash=none}

\def\tikzname{Ti\emph{k}Z}


% stolen from minted.dtx
\newenvironment{exampletwoup}
  {\VerbatimEnvironment
   \begin{VerbatimOut}{example.out}}
  {\end{VerbatimOut}
   \setlength{\parindent}{0pt}
   \fbox{\begin{tabular}{l| l}
   \begin{minipage}{0.55\linewidth}
     \inputminted[fontsize=\small,resetmargins]{latex}{example.out}
   \end{minipage} &
   \begin{minipage}{0.35\linewidth}
     \input{example.out}
   \end{minipage}
   \end{tabular}}}

\newenvironment{exampletwouptiny}
  {\VerbatimEnvironment
   \begin{VerbatimOut}{example.out}}
  {\end{VerbatimOut}
   \setlength{\parindent}{0pt}
   \fbox{\begin{tabular}{l|l}
   \begin{minipage}{0.55\linewidth}
     \inputminted[fontsize=\scriptsize,resetmargins]{latex}{example.out}
   \end{minipage} &
   \begin{minipage}{0.35\linewidth}
     \setlength{\parskip}{6pt plus 1pt minus 1pt}%
     \raggedright\scriptsize\input{example.out}
   \end{minipage}
   \end{tabular}}}

% ******************************** Meta-data ***********************************
\mode<presentation>
{
  \usetheme{Madrid}
  \setbeamercovered{transparent}
}


\usepackage{caption}
\captionsetup{font=scriptsize, labelfont=scriptsize, justification=centering}

\title{Writing papers and thesis using \LaTeX2e}

\author {Krishna Kumar \inst{*}\thanks{krishnak@utexas.edu} }

\institute[ UT Austin] % (optional, but mostly needed)
{
  \inst{1}%
  Cockrell School of Engineering\\
  The University of Texas at Austin
}

\pgfdeclareimage[height=0.2cm]{university-logo}{figs/ut.png}

\date[LaTeX Course]{\LaTeX for Beginners}
% Delete this, if you do not want the table of contents to pop up at
% the beginning of each subsection:
\AtBeginSection[]
{
  \begin{frame}<beamer>{Outline}
    \tableofcontents[currentsection,currentsubsection]
  \end{frame}
}

% If you wish to uncover everything in a step-wise fashion, uncomment
% the following command: 

% \beamerdefaultoverlayspecification{<+->}

\subtitle{Part II: Writing papers and thesis using \LaTeX}
%***************************** Title page **************************************
\begin{document}
\begin{frame}
  \titlepage
\end{frame}
%*******************************************************************************
%**************************** Introduction *************************************
%*******************************************************************************
\section{Graphics}

%*******************************************************************************
%******************************* Frame *****************************************
%*******************************************************************************
\begin{frame}[fragile]{Figures}
\begin{itemize}
\item \LaTeX can be easily extended using a \emph{package} to typeset images.

\item To use graphics in your \LaTeX document use \cmdbs{usepackage\{graphicx\}}

\item Always use relative scaling to specify the width of the figure, i.e., 

\cmmd{[width = 0.75\bs textwidth]}

\item Never ever use absolute values to scale your images!

\item Set either the width or the height of the image. Or use \cmmd{scale}

\end{itemize}
\begin{exampletwoup}
\begin{figure}
\includegraphics[width=0.75\textwidth]
                           {figs/minion}
\end{figure}
\end{exampletwoup}
\end{frame}


%*******************************************************************************
%******************************* Frame *****************************************
%*******************************************************************************
\begin{frame}[fragile]{Figures}
\begin{itemize}
\item For captioning a figure, you can use \cmdbs{usepackage\{caption\}}

\item tweak the location, label, separator: \cmmd{[labelsep=space, 
tableposition=top]\{caption\}}

\item I prefer to centre the figure. To do that use \cmdbs{centering}

\item You can use \cmmd{\~\bs cref\{fig:minion\}} to cross reference the figure.

\end{itemize}
\begin{exampletwoup}
\begin{figure}
\centering
\includegraphics[width=0.75\textwidth]
                           {figs/minion}
\caption[Minion]{
	Dave the Minion from Despicable Me!}
\label{fig:minion}
\end{figure}
\end{exampletwoup}
\end{frame}

%*******************************************************************************
%******************************* Frame *****************************************
%*******************************************************************************
\begin{frame}[fragile]{\bs begin[option]\{figure\}}
\renewcommand{\arraystretch}{1.5}
\begin{tabularx}{\textwidth}{l X}
\toprule
\textbf{Parameter} &	\textbf{Position} \\ \midrule
h &	Place the float here, i.e., approximately at the same point it occurs in 
the source text (however, not exactly at the spot) \\
t &	Position at the top of the page. \\
b &	Position at the bottom of the page. \\
p &	Put on a special page for floats only. \\
! &	Override internal parameters LaTeX uses for determining "good" float 
positions. \\
H &	Places the float at precisely the location in the LaTeX code. Requires the 
float package. This is somewhat equivalent to h! \\ \bottomrule
\end{tabularx}
\end{frame}

%*******************************************************************************
%******************************* Frame *****************************************
%*******************************************************************************
\begin{frame}[fragile]{Subcaption}

I can cite Wall-E (see~\cref{fig:WallE}) and Minions in despicable me 
(\cref{fig:Minnion}).~\Cref{fig:animations} lets me cite the whole figure.

\begin{exampletwouptiny}
\begin{figure}
  \centering
  \begin{subfigure}[b]{0.3\textwidth}
    \includegraphics[width=\textwidth]
			    {figs/TomandJerry}
    \caption{Tom and Jerry}
    \label{fig:TomJerry}   
  \end{subfigure}             
  \begin{subfigure}[b]{0.3\textwidth}
    \includegraphics[width=\textwidth]{figs/WallE}
    \caption{Wall-E}
    \label{fig:WallE}
  \end{subfigure}             
  \begin{subfigure}[b]{0.3\textwidth}
    \includegraphics[width=\textwidth]{figs/minion}
    \caption{Minions}
    \label{fig:Minnion}
  \end{subfigure}
  \caption{Best Animations}
  \label{fig:animations}
\end{figure}
\end{exampletwouptiny}
\end{frame}


%*******************************************************************************
%******************************* Frame *****************************************
%*******************************************************************************
\begin{frame}[fragile]{Exercise 6: Pictures}

\begin{center}
\fbox{\href{\wlnewdoc{Ex6-Pictures/Ex6-Pictures.tex}}{%
Click to open this exercise in \wllogo{}}}
\end{center}

\begin{itemize}
\item Please upload the figures in \cmmd{figs/} to the write\LaTeX project. or 
point to raw URLs on github.
\item Format these the King's college picture. Change it's location: (top, 
bottom, here, new page)
\item Arrange Tom, WallE and Dave inside a single figure environment 
vertically.
\item Use Fig. to refer to figures
\item Create a list of figures.
\end{itemize}

\begin{center}
%
\fbox{\href{\wlnewdoc{Ex6-Pictures/Ex6-Pictures-solution.tex}}{%
Click to open my solution}}.
\end{center}

\end{frame}


%*******************************************************************************
%******************************* Frame *****************************************
%*******************************************************************************
\subsection{Tables}
\begin{frame}[fragile]{Tables}
\begin{itemize}
\item Tables in \LaTeX{} take some getting used to.
\item The argument specifies column alignment --- \textbf{l}eft, 
\textbf{r}ight, \textbf{r}ight.
\begin{exampletwouptiny}
\begin{tabular}{lcr}
Item   & Qty & Unit \$ \\
Widget & 1   & 199.99  \\
Gadget & 2   & 399.99  \\
Cable  & 3   & 19.99   \\
\end{tabular}
\end{exampletwouptiny}
\item Don't use vertical lines, it's ugly. Use \cmdbegin{booktabs} to create 
horizontal lines. Never use \cmdbs{hline}
\begin{exampletwouptiny}
\begin{table}[h]
\caption{Cost}
\begin{tabular}{lrr} \toprule
Item   & Qty & Unit \$ \\ \midrule
Widget & 1   & 199.99  \\
Gadget & 2   & 399.99  \\
Cable  & 3   & 19.99   \\ \bottomrule
\end{tabular}
\label{t:cost}
\end{table}
\end{exampletwouptiny}
\item Use an ampersand \keys{\&} to separate columns and a double 
backslash \keys{\bs}\keys{\bs} to start a new row (like in 
the \bftt{align*} environment that we saw in part 1).
\end{itemize}
\end{frame}

\begin{frame}{Table Environment}

\begin{table}
\begin{tabularx}{0.9\textwidth}{l X} \toprule
\textbf{Option} & \textbf{Description} \\ \midrule
l  &	left-justified column \\
c  &	centered column  \\
r  &	right-justified column  \\
p\{`width'\} &	paragraph column with text vertically aligned at the top  \\
m\{`width'\} & 	paragraph column with text vertically aligned in the middle \\
b\{`width'\} &	paragraph column with text vertically aligned at the bottom  \\
\cmdbs{\&} & 	column separator \\
\cmmd{\bs \bs} &	start new row (additional space may be specified \\
\cmdbs{cmidrule\{i-j\}} & 	partial horizontal line beginning in column i and 
ending in column j \\ \bottomrule
\end{tabularx}
\end{table}
\end{frame}

%*******************************************************************************
%******************************* Frame *****************************************
%*******************************************************************************
\begin{frame}[fragile]{Exercise 7: Tables}

\begin{center}
\fbox{\href{\wlnewdoc{Ex7-Tables/Ex7-Tables.tex}}{%
Click to open this exercise in \wllogo{}}}
\end{center}

\begin{itemize}
\item Use \cmmd{tabularx} package for tables with paragraph text.

\item Never use \cmdbs{hline} or \cmdbs{cline}, use \cmdbs{toprule}, 
\cmdbs{midrule}, \cmdbs{bottomrule} and \cmdbs{cmidrule\{i-j\}}

\item Visual table editor: 
\href{http://truben.no/table/}{http://truben.no/table/}
\end{itemize}

\begin{center}
%
\fbox{\href{\wlnewdoc{Ex7-Tables/Ex7-Tables-solution.tex}}{%
Click to open my solution}}.
\end{center}

\end{frame}



%*******************************************************************************
%******************************* Frame *****************************************
%*******************************************************************************
\subsection{bib\TeX}
\begin{frame}[fragile]{\insertsubsection{} 1}
\begin{itemize}
\item Put your references in a \bftt{.bib} file in `bibtex' database format:
\inputminted[fontsize=\scriptsize,frame=single]{latex}{bib-example.bib}
\item Most reference managers can export to bibtex format.
\end{itemize}
\end{frame}


%*******************************************************************************
%******************************* Frame *****************************************
%*******************************************************************************
\begin{frame}[fragile]{\insertsubsection{} 2}
\begin{itemize}
\item Each entry in the \bftt{.bib} file has a \emph{key} that you can use to
reference it in the document. For example, \bftt{Jacobson1999Towards} is the 
key for this article:
\begin{minted}[fontsize=\small,frame=single]{latex}
@Article{Jacobson1999Towards,
  author = {Van Jacobson},
  ...
}
\end{minted}
\item It's a good idea to use a key based on the name, year and title.
\item \LaTeX{} can automatically format your in-text citations and generate a
list of references; it knows most standard styles, and you can design your own.
\end{itemize}
\end{frame}


%*******************************************************************************
%******************************* Frame *****************************************
%*******************************************************************************
\begin{frame}[fragile]{\insertsubsection{} 3}
\begin{itemize}
\item Use the \bftt{natbib} package\footnote{There is a new package with more
  features named \bftt{biblatex} but most of the articles templates still use
  \bftt{natbib}.} with \cmdbs{citet} and \cmdbs{citep}.
\item Reference \cmdbs{bibliography} at the end, and specify a 
\cmdbs{bibliographystyle}.
\end{itemize}
\begin{minipage}{0.55\linewidth}
\inputminted[fontsize=\scriptsize,frame=single,resetmargins]{latex}%
  {bib-example.tex}
\end{minipage}
\begin{minipage}{0.35\linewidth}
\includegraphics[width=\textwidth,clip,trim=1.8in 5in 1.8in 
1in]{bib-example.pdf}
\end{minipage}
\end{frame}

%*******************************************************************************
%******************************* Frame *****************************************
%*******************************************************************************
\begin{frame}[fragile]{Exercise 8: Bibliography}

\begin{center}
\fbox{\href{\wlnewdoc{Ex7-Tables/Ex7-Tables.tex}}{%
Click to open this exercise in \wllogo{}}}
\end{center}

\begin{itemize}
\item Use \cmmd{tabularx} package for tables with paragraph text.

\item Never use \cmdbs{hline} or \cmdbs{cline}, use \cmdbs{toprule}, 
\cmdbs{midrule}, \cmdbs{bottomrule} and \cmdbs{cmidrule\{i-j\}}

\item Visual table editor: 
\href{http://truben.no/table/}{http://truben.no/table/}
\end{itemize}

\begin{center}
%
\fbox{\href{\wlnewdoc{Ex7-Tables/Ex7-Tables-solution.tex}}{%
Click to open my solution}}.
\end{center}

\end{frame}


%*******************************************************************************
%******************************* Frame *****************************************
%*******************************************************************************
\begin{frame}[fragile]{Exercise 8: Formatting a Paper}

\begin{center}
\fbox{\href{\wlnewdoc{Ex8-Paper/Ex8-Paper.tex}}{%
Click to open this exercise in \wllogo{}}}
\end{center}

\begin{center}
\fbox{\href{\wlnewdoc{Ex8-Paper/paper.bib}}{%
Upload paper.bib to \wllogo{}}}
\end{center}


\begin{center}
\fbox{\href{\wlnewdoc{Ex8-Paper/figure.pdf}}{%
Upload figure.pdf to \wllogo{}}}
\end{center}

\begin{itemize}
\item eference styles: 
\href{http://sites.stat.psu.edu/~surajit/present/bib.htm}{http://sites.stat.psu.edu/~surajit/present/bib.htm}

\item Bibliography style previews: 
\href{http://nodonn.tipido.net/bibstyle.php}{http://nodonn.tipido.net/bibstyle.php}
\end{itemize}

\begin{center}
%
\fbox{\href{\wlnewdoc{Ex8-Paper/Ex8-Paper-solution.tex}}{%
Click to open my solution}}.
\end{center}

\end{frame}


%*******************************************************************************
%******************************* Frame *****************************************
%*******************************************************************************
\begin{frame}[fragile]{Exercise 8: Bibliography}

\begin{center}
\fbox{\href{\wlnewdoc{Ex7-Tables/Ex7-Tables.tex}}{%
Click to open this exercise in \wllogo{}}}
\end{center}

\begin{itemize}
\item Use \cmmd{tabularx} package for tables with paragraph text.

\item Never use \cmdbs{hline} or \cmdbs{cline}, use \cmdbs{toprule}, 
\cmdbs{midrule}, \cmdbs{bottomrule} and \cmdbs{cmidrule\{i-j\}}

\item Visual table editor: 
\href{http://truben.no/table/}{http://truben.no/table/}
\end{itemize}

\begin{center}
%
\fbox{\href{\wlnewdoc{Ex7-Tables/Ex7-Tables-solution.tex}}{%
Click to open my solution}}.
\end{center}

\end{frame}


%*******************************************************************************
%******************************* Frame *****************************************
%*******************************************************************************
\begin{frame}[fragile]{Thesis Template}

\begin{center}
\fbox{\href{https://www.writelatex.com/templates/phd-thesis-template-for-cambridge-university-engineering-department-cued/kgfqybfnqkdf}{%
Click to open in \wllogo{}}}
\end{center}

\begin{center}
\fbox{\href{https://www.sharelatex.com/templates/545260fcca5682183a610825}{%
Click to open in Share\LaTeX}}
\end{center}


\end{frame}%
\end{document}
