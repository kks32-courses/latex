\usepackage{amsfonts}
\usepackage{amsmath}
\usepackage{amssymb}
\usepackage{mathptmx}
\usepackage{graphicx}
\usepackage{subcaption}
\usepackage{cleveref}
\usepackage{color}
\usepackage{minted}
\usepackage{hyperref}
\usepackage{multicol}
\usepackage{multirow}
\usepackage{tabularx}
\usepackage{booktabs}
\usepackage{menukeys}

% Stolen from John Miller's LaTeX course
\newcommand{\bftt}[1]{\textbf{\texttt{#1}}}
\newcommand{\comment}[1]{{\color[HTML]{008080}\textit{\textbf{\texttt{#1}}}}}
\newcommand{\cmmd}[1]{{\color[HTML]{008000}\bftt{#1}}}
\newcommand{\bs}{$\backslash$}
\newcommand{\cmdbs}[1]{\cmmd{\bs#1}}
\newcommand{\lb}{{\char'173}}% Left brackets -> {
\newcommand{\rb}{{\char'175}}% Right brackets -> }
\newcommand{\cmdbegin}[1]{\cmdbs{begin\lb}\bftt{#1}\cmmd{\rb}}
\newcommand{\cmdend}[1]{\cmdbs{end\lb}\bftt{#1}\cmmd{\rb}}



%\newcommand{\wllogo}{\textbf{write\textrm{\LaTeX}}}

% this is where the example source files are loaded from
% do not include a trailing slash
\newcommand{\wllogo}{\textbf{Overleaf}}
\newcommand{\fileuri}{https://raw.githubusercontent.com/kks32-courses/latex/master/exercises/}
\newcommand{\wlserver}{https://www.overleaf.com}
\newcommand{\wlnewdoc}[1]{\wlserver/docs?snip\_uri=\fileuri#1\&splash=none}
\newcommand{\wlnewdocarray}[1]{\wlserver/docs?snip\_uri[]=\fileuri#1}
\newcommand{\wlmorearray}[1]{&snip\_uri[]=\fileuri#1}

\def\tikzname{Ti\emph{k}Z}


% stolen from minted.dtx
\newenvironment{exampletwoup}
  {\VerbatimEnvironment
   \begin{VerbatimOut}{example.out}}
  {\end{VerbatimOut}
   \setlength{\parindent}{0pt}
   \fbox{\begin{tabular}{l| l}
   \begin{minipage}{0.55\linewidth}
     \inputminted[fontsize=\small,resetmargins]{latex}{example.out}
   \end{minipage} &
   \begin{minipage}{0.35\linewidth}
     \input{example.out}
   \end{minipage}
   \end{tabular}}}

\newenvironment{exampletwouptiny}
  {\VerbatimEnvironment
   \begin{VerbatimOut}{example.out}}
  {\end{VerbatimOut}
   \setlength{\parindent}{0pt}
   \fbox{\begin{tabular}{l|l}
   \begin{minipage}{0.55\linewidth}
     \inputminted[fontsize=\scriptsize,resetmargins]{latex}{example.out}
   \end{minipage} &
   \begin{minipage}{0.35\linewidth}
     \setlength{\parskip}{6pt plus 1pt minus 1pt}%
     \raggedright\scriptsize\input{example.out}
   \end{minipage}
   \end{tabular}}}


\newenvironment{exampletwouptinyfifty}
  {\VerbatimEnvironment
   \begin{VerbatimOut}{example.out}}
  {\end{VerbatimOut}
   \setlength{\parindent}{0pt}
   \fbox{\begin{tabular}{l|l}
   \begin{minipage}{0.45\textwidth}
     \inputminted[fontsize=\scriptsize,resetmargins]{latex}{example.out}
   \end{minipage} &
   \begin{minipage}{0.45\textwidth}
     \setlength{\parskip}{6pt plus 1pt minus 1pt}%
     \raggedright\scriptsize\input{example.out}
   \end{minipage}
   \end{tabular}}}

% ******************************** Meta-data ***********************************
\mode<presentation> {
	
	\usetheme{Madrid}
	
	% Burnt orange
	\definecolor{burntorange}{rgb}{0.8, 0.33, 0.0}
	\colorlet{beamer@blendedblue}{burntorange}
	% Pale yellow
	\definecolor{paleyellow}{rgb}{1.0, 1.0, 0.953}
	\setbeamercolor{background canvas}{bg=paleyellow}
	% Secondary and tertiary palett
	\setbeamercolor*{palette secondary}{use=structure,fg=white,bg=burntorange!80!black}
	\setbeamercolor*{palette tertiary}{use=structure,fg=white,bg=burntorange!60!black}
	
	% To remove the footer line in all slides uncomment this line
	%\setbeamertemplate{footline}
	% To replace the footer line in all slides with a simple slide count uncomment this line
	%\setbeamertemplate{footline}[page number]
	
	% To remove the navigation symbols from the bottom of all slides uncomment this line
	%\setbeamertemplate{navigation symbols}{}
}

\usepackage{caption}
\captionsetup{font=scriptsize, labelfont=scriptsize, justification=centering}

\title{Writing papers and thesis using \LaTeX2e}

\author {Krishna Kumar \inst{*}\thanks{krishnak@utexas.edu} }

\institute[UT Austin] % (optional, but mostly needed)
{
  \inst{1}%
  Cockrell School of Engineering\\
  The University of Texas at Austin
}

\pgfdeclareimage[height=0.2cm]{university-logo}{figs/ut.png}

\date[LaTeX Course]{\LaTeX for Beginners}
% Delete this, if you do not want the table of contents to pop up at
% the beginning of each subsection:
\AtBeginSection[]
{
  \begin{frame}<beamer>{Outline}
    \tableofcontents[currentsection,currentsubsection]
  \end{frame}
}

% If you wish to uncover everything in a step-wise fashion, uncomment
% the following command: 

% \beamerdefaultoverlayspecification{<+->}
