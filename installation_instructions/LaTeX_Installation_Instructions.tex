\documentclass[times,twoside,11pt]{article}

\usepackage[pdftex,paper=a4paper,hmarginratio=1:1,          
vmarginratio=1:1]{geometry}

% ************************ URL Package and Definition **************************
\RequirePackage{url}
\renewcommand{\UrlFont}{\normalsize}
\urlstyle{leo}


% For PDF Online version
\usepackage[pdftex, breaklinks=true, linktocpage,
  colorlinks=true, linkcolor=blue, urlcolor=blue, 
  citecolor=blue, anchorcolor=green]{hyperref}

% ************************** To copy ligatures ******************************* %
\usepackage[utf8]{inputenc}
\usepackage[T1]{fontenc}

\input{glyphtounicode}
\pdfglyphtounicode{f_f}{FB00}
\pdfglyphtounicode{f_i}{FB01}
\pdfglyphtounicode{f_l}{FB02}
\pdfglyphtounicode{f_f_i}{FB03}
\pdfglyphtounicode{f_f_l}{FB04}
\pdfgentounicode=1


% ********************************** Listings **********************************
\usepackage{listings}
\usepackage{color}
\usepackage{xcolor}

\definecolor{light-gray}{gray}{0.95}
\lstdefinestyle{shell}{
  backgroundcolor=\color{light-gray},
  belowcaptionskip=1\baselineskip,
  breaklines=true,
  xleftmargin=\parindent,
  language=sh,
  showstringspaces=false,
  keywordstyle=\bfseries\color{black},
  commentstyle=\itshape\color{black},
  identifierstyle=\color{black},
  stringstyle=\color{black},
  morekeywords={sudo, apt-get, yum, zypper, mount},
  columns=flexible,texcl,
  mathescape=true,
}


% **************************** Title and Info ******************************** %

\title{\LaTeXe  Installation Instructions}
\author{Krishna Kumar\thanks{kks32@cam.ac.uk}}
\date{}

\begin{document}
\maketitle

\section*{Windows OS}
\subsection*{\TeX distribtion}
\subsubsection*{\TeX Live (Recommended)}
\begin{enumerate}
\item	Download the \TeX Live ISO ($\sim 2.5$~GB) from 
\url{http://anorien.csc.warwick.ac.uk/mirrors/CTAN/systems/texlive/Images/texlive2015.iso}
\item	Download WinCDEmu (if you don't have a virtual drive) from 
\url{http://wincdemu.sysprogs.org/download/}
\item	To install Windows CD Emulator follow the instructions at 
\url{http://wincdemu.sysprogs.org/tutorials/install/}
\item	Right click the iso and mount it using the WinCDEmu as shown in 
\url{http://wincdemu.sysprogs.org/tutorials/mount/}
\item	Open your virtual drive and run \verb|setup.pl|
\end{enumerate}

\subsubsection*{Mik\TeX}
\begin{enumerate}
\item	Download Basic-MikTex 2.9 (32bit or 64bit) from 
\url{http://miktex.org/download}
\item	Run the installer 
\url{http://docs.miktex.org/2.9/manual/ch02s02.html}
\item	To add a new package go to Start >> All Programs >> MikTex 2.9 >> 
Maintenance (Admin) and choose Package Manager
\item	Select or search for packages to install
\end{enumerate}


\subsection*{TexStudio (\TeX Editor)}
\begin{enumerate}
\item	Download TexStudio 2.10 from 
\url{http://texstudio.sourceforge.net/\#downloads} 
\item	Run the installer
\end{enumerate}

\section*{Mac OS X}
\subsection*{MacTex (Install \TeX Live 2015)}
\begin{enumerate}
\item	
	Download the file from 
	\url{http://mirror.ctan.org/systems/mac/mactex/MacTeX.pkg}
\item	
	Extract and double click to run the installer. It does the entire 
	configuration, sit back and relax.
\end{enumerate}

\subsection*{TexStudio (\TeX Editor)}
\begin{enumerate}
\item	
	Download \TeX Studio 2.10 from 
	\url{http://texstudio.sourceforge.net/\#downloads}
 
\item
	Extract and Start
\end{enumerate}


\section*{Unix/Linux:}

\subsection*{Installation using Linux packages} 
\subsubsection*{Fedora/RedHat/CentOS:}
\begin{lstlisting}[style=shell]
sudo yum install texlive 
sudo yum install psutils 
\end{lstlisting}


\subsubsection*{SUSE:}
\begin{lstlisting}[style=shell]
sudo zypper install texlive
\end{lstlisting}


\subsubsection*{Debian/Ubuntu:}
\begin{lstlisting}[style=shell] 
sudo apt-get install texlive texlive-latex-extra 
sudo apt-get install psutils
\end{lstlisting}

\subsection*{Direct installation (latest version)}
\subsubsection*{Getting the distribution}
\begin{enumerate}
\item	
	\TeX Live can be downloaded from 
	\url{https://www.tug.org/texlive/acquire.html}. You might require 
	wget to download through proxies.

\item
	\TeX Live is provided by most operating system you can use (rpm, 
	apt-get or yum) to get \TeX Live distributions (note: usually it's the old 
	\TeX Live) 
\end{enumerate}

\subsubsection*{Installation}
\begin{enumerate}
\item	
	Mount the ISO file in the mnt directory
	\begin{lstlisting}[style=shell]
	mount -t iso9660 -o ro,loop,noauto /your/texlive2015.iso /mnt
	\end{lstlisting}

\item
	Get wget on your OS (use rpm, apt-get or yum install)
\item
	Run the installer script install-tl.

\begin{lstlisting}[style=shell]
cd /your/download/directory
./install-tl
\end{lstlisting}

\item
	Enter command `i' for installation

\item
	Post-Installation configuration (Environment variables for Unix): 
	\url{http://www.tug.org/texlive/doc/texlive-en/texlive-en.html\#x1-320003.4.1}
\item	
	Set the path for the directory of \TeX Live binaries in your 
	\verb|~/.bashrc| 
	file:

\paragraph{64Bit}

\begin{verbatim}
PATH=/usr/local/texlive/2015/bin/x86_64-linux:$PATH; export PATH
MANPATH=/usr/local/texlive/2015/texmf/doc/man:$MANPATH; export MANPATH 
INFOPATH=/usr/local/texlive/2015/texmf/doc/info:$INFOPATH; export INFOPATH
\end{verbatim}

\paragraph{32Bit}

\begin{verbatim}
PATH=/usr/local/texlive/2015/bin/i386-linux:$PATH; export PATH 
MANPATH=/usr/local/texlive/2015/texmf/doc/man:$MANPATH; export MANPATH 
INFOPATH=/usr/local/texlive/2015/texmf/doc/info:$INFOPATH; export INFOPATH
\end{verbatim}

\end{enumerate}

\end{document}
